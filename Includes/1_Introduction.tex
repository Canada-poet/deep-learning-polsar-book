\chapter{Introduction}


Remotely sensed images acquired by space-borne or airborne sensors have revolutionized earth observation allowing an unprecedented ability to image and observe large areas of the planet's surface with high temporal repetitivity. This has opened new avenues for the monitoring of dynamic earth processes. Long-term earth observation missions have ensured data continuity over decades allowing the characterization of natural and anthropocentric processes and changes. However, in order for the images to be used for these applications, they must be processed so that the information of interest is brought out. Traditionally thematic maps have been produced by a field survey of the region to be mapped. This technique, while accurate, is time-consuming and expensive; Prohibitively so for large areas. The use of remote sensing techniques allows for the production of small to medium scale thematic maps (encompassing continents or even world maps) quickly and economically with a fair degree of accuracy~\cite{gong2013finer}. 

Classification of remote sensing data involves grouping of pixels into homogeneous groups, usually representing a type of land cover. This allows the generation of thematic maps. These maps depict various land covers depending on the target application. For example, an agriculture map would typically denote various crop types in an area. Thematic maps play a major role in the large-scale monitoring of several geophysical parameters. Forest cover maps are used to quantify and control deforestation, sea-ice maps are crucial for monitoring the motion of ice-bergs thus safeguarding navigation channels, agriculture maps are often used for yield prediction of food grains, flood area maps help in the minimization of loss to property and mitigation of damage due to flood inundation, urban area maps help study the expansion of human settlement over decades etc. Many of these maps, especially those used for disaster management or the tracking of fast-changing geo-dynamics, must be generated rapidly with a fair degree of accuracy. This information helps shape governance policies, aid legislative decision-making processes and help generate a better understanding of the world around us. 


Classification of remotely sensed images entails the automated identification of targets like buildings, cars, armament etc. in acquired scenes relying on training libraries, previously identified targets or other target descriptors. The all-weather, day-night imaging capability of a Synthetic Aperture Radar (SAR) system is highly desirable for defense, security and monitoring applications, where time-critical imaging is required. Unmanned aerial vehicle (UAV) borne SAR systems are being operationally used for information gathering. A peripatetic platform like this is capable of high repetitivity and frequent re-visits, thus it generates a high volume of data. 

In such a scenario, the ability to extract information from the datasets in an automated manner is essential. However, object recognition and classification in SAR images is not without its challenges. SAR images suffer from geometric distortions and speckle-like noise patterns which are inherent to the process of SAR image formation. Polarimetric SAR (PolSAR) images contain more information than single or dual channel SAR, but it is important to develop techniques to utilize this information entirely.  

Automation in classification and identification tasks, or at least part-automation, can help completely utilize the potential of the large data volumes. Machine learning algorithms can help collate information from the data, which can eventually lead to the extraction of knowledge from the information. Learning machines are designed to tune their internal parameters according to the input data to improve their efficiency and/or accuracy. Machine learning algorithms are able to utilize the extensive data quantities. However, traditional machine learning algorithms usually need specially designed input features, either from $a-priori$ information about the distribution of the data, or domain knowledge, for a particular application, making complete automation prohibitive. Also, beyond a point, increase in the amount of available training data does not lead to a corresponding improvement in task efficacy~\cite{bengio2009learning}.

Deep Learning is a new approach based on the workings of the human cognitive system pioneered in 2006 by ~\cite{hinton2006fast}. The principle behind deep learning is to distribute a complex task into simpler subtasks. Different parts of the network can then be responsible for each sub-task, eventually being connected cohesively to solve the original task. It has shown superior performance as compared to 'shallow' approaches in various classification and recognition challenges and is rapidly being applied to diverse fields like image classification, object recognition, speech and language processing, etc. Deep Learning techniques greatly benefit from increased training information, more so than other contemporary methods, making them a good fit for remote sensing applications in which high volumes of similar data is collected.  

%This thesis aims to overcome some of the challenges using deep learning based approaches in radar remote sensing, and collectively enhance earth observation applications.

\section{Motivation and Highlights}

Optical remote sensing has been primarily used for earth observation, monitoring and disaster management applications. However, being a passive sensor, it's limited to operating in the presence of solar irradiation. Further, it is impeded by the presence of cloud cover. SAR, on the other hand, is an active imaging technique and has an advantage that it is able to operate independently of ambient illumination. It is also insensitive to the weather due to its capability of penetrating cloud cover. SAR is sensitive to both the wavelength scale geometrical and dielectric properties of the target, allowing it to ascertain information about the target that is not possible with reflectance alone. 

PolSAR is an advancement over SAR imaging, which uses a combination of transmit and receive polarizations in quadrature. Thus, it is sensitive to the wavelength-scale structure of the target and has better discrimination potential. To be able to exploit this increased target information fully, and to efficiently handle the increasing volume of data being collected by present and planned satellite constellations, automated analysis techniques must be developed. 

Machine learning algorithms are a reasonable choice for this task, however, hand-crafted features need to be designed from domain knowledge and require expert intervention, making it time-consuming and prohibitive for large data volumes. Additionally, the mathematical representation or polarimetric data and pertinent models is more involved, requiring the use of abstract complex-valued matrix algebra. This makes developing and using theoretical physical models a cognitively challenging task. The aim of deep learning is to develop a simple feature representation internally and automatically without the need for hand-crafted or a specially designed input space. This makes Deep Learning techniques especially synergistic for PolSAR applications, potentially helping replace the challenges of optimization and model building with data intensive self-training approached. However, the use of indiscriminate machine learning techniques can be problematic since there is no guarantee that the learned model makes sense from a physical standpoint. It is advantageous to explore the possibilities of enhancing Earth observation applications of PolSAR data using novel deep learning techniques, however, with constrains and parameterizations developed in reverence to and limited by the laws of physics and electromagnetic interactions.


%\section{Highlights}
In this book, novel deep learning algorithms and architectures are detailed for various earth observation applications using fully polarimetric SAR data based on scattering physics principles.
The methodologies have been developed and verified using airborne L-band UAVSAR, airborne C-, L-, P-band AIRSAR, spaceborne C-band RADARSAT-2 and spaceborne L-band ALOS-2 datasets. Applications have been demonstrated in the field of agricultural monitoring, urban classification, change detection and disaster response. 

The following highlights are detailed in the course of this book:

\begin{enumerate}
\item Auto-Encoders have been demonstrated in finding an efficient sparse feature-space representation of PolSAR data.
\item Scattering physics has been incorporated in an auto-encoders architecture using a novel technique for data-augmentation, which improves generalization capacity. 
\item A novel deep-learning framework has been developed that outperformed state of the art techniques in the classification of urban areas, especially those not perpendicular to the line of sight of the radar.
\item Development of a novel semi-supervised technique that allows change detection in urban areas with very few training points using a region growing technique in the auto-encoder's feature space. 
\item A novel tensorization technique in conjunction with a spatial auto-encoder has been demonstrated that is able to combine information from multi-frequency bands to improve classification of natural areas. 
\end{enumerate}


%\section{Motivation}
%\section{Research Objectives}
%\section{Thesis Outline}