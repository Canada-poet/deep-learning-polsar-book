\chapter{Summary and Future Work}

A survey of developments in deep learning to remote-sensing and SAR classification techniques and a review of the state of the art has been presented. We have explored the application of deep learning to address the challenges associated with SAR classification, especially polarimetric SAR. The problem of orientation compensation for rotated targets was identified as a major source of misclassification of urban structures, which could be potentially be alleviated. We applied incremental rotations to the PolSAR data to generate synthetic data-sets. Subsequently, using an auto-encoder, we learn a sparse representation of the entire generated data-set. The feature space thus incorporates the complete target information, including the effect of rotation. These features are then applied to an MLP network which performs the final classification. The resultant classification is significantly better than the commonly used classification algorithms. The output of this, the urban classification can be used in the study of urban sprawl with temporal data. 

Further, a weakly supervised change detection technique was developed. 
Often sufficient ground truth data is not available to use traditional supervised machine learning techniques. A novel Deep Learning based weakly-supervised framework for urban change detection using multi-temporal polarimetric SAR data is developed. A modified unsupervised stacked auto-encoder stage is used to learn an efficient representation of the multi-temporal polarimetric information. Then a label aggregation is performed in the feature space before classification by a multi-layer perceptron. The proposed methodology is validated on an L-band UAVSAR dataset acquired over Los Angeles, CA and performs accurately and effectively with a low false alarm rate.  

Subsequently, a novel tensorization framework in conjunction with an ANN algorithm is proposed to leverage the increased information content present in multi-frequency PolSAR data. The Kronecker product is used to combine information from multi-frequency PolSAR data. The resultant combination is efficiently utilized by an ANN. This is demonstrated on a C-, L-, P-band data-set for crop and forest classification. Pairwise tensor combinations of frequency bands (CL, CP, LP ) performs better than the individual band, which in turn, is outperformed by the tensorized triplet (CLP ). This shows that the tensorized combination of frequency bands have increased information content, which leads to a better classification performance when used in conjunction with an ANN architecture. Moreover, the CLP combination outperforms simple band
augmentation (CLP + ). In the future, this technique can be exploited for multi-temporal and multi-incidence datasets from advanced sensors for improved classification performance.

%\section{Conclusion}
%\label {sec:conclusion}
Finally, a novel Poincar\'e sphere and AE based algorithm for generation of snow-cover maps from full and hybrid polarimetric data have been demonstrated. The reconstruction by the auto-encoder of the original parameters are shown and found to be similar, with an increase in dynamic range. The representational layer of the AE is subsequently extracted and given as input to an MLP layer for classification. This representation has improved separability and leads to a less complex final classification layer, and improved snow-cover classification performance. 
In the future, spatial inputs can be exploited to the AE to account for inter-relations between the pixels. The normalized DSM can also be included to allow incorporation of the topographic features which affects the returned backscatter. 
%In effect, it should be possible to learn the various effects caused by the undulations in terrain and compensate for it in the classification process. 
Moreover, an optimal transmission polarization can be determined, which can help further improve separability. 

In the future is it intended to expand the functionality of the developed techniques with the following additional concepts:

\subsubsection*{Transfer Learning}
To make the classification technique more generally applicable to large volumes of data, it is desirable that supervision is minimized in the process. This can be done by implementing a unsupervised technique or by transferring the knowledge acquired in the training of one image to the subsequent unseen images. The process of transferring the learning can usually result in a higher overall accuracy on the entire set of images, given that the statistics of the underlying data / scene vary only marginally. This is usually true for images acquired in the same season, using the same or compatible sensor and similar geographic region. Sensors like Sentinel have a set collection strategy that allows them to collect temporally and spatially consistent regular SAR datasets globally. This trend will be continued by future SAR missions like the RCM and NISAR, leading to a large volume of data compliant to the aforementioned criterion. A transfer learning based semi-supervised technique will greatly help extract information from the data. It is seen that the method shows a reasonable transfer of learning, and the performance can be improved by fine tuning.



\subsubsection*{Variational Auto-Encoders}
The Variational Auto-Encoder (VAE) is a generative model that uses bayesian inference principles to create examples based on the samples it has been trained with under probabilistic constrains. It uses $a-priori$ information of the underlying probability distribution of the data to sample apply this constraint and thus embed a latent distribution model in the learned representation. Since the statistical properties of SAR data and speckle is well known, with the coherent returns in a multilooked scene following a Gamma distribution and the speckle pattern following an Inverse Gamma distribution. The VAE can be used to generate radar return samples, under the scattering physics constrains that can drastically improve generalization capabilities of the network. Such a method would find application in urban classification to improve performance in the areas oriented away from the radar line of sight. 
